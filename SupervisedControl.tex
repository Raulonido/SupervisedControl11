\documentclass[10pt,a4paper]{article}
\usepackage[latin1]{inputenc}
\usepackage{amsmath}
\usepackage{amsfonts}
\usepackage{amssymb}
\usepackage{graphicx}
\author{Raul Nistal\thanks{Raul Nistal works as researcher at Department of Electricity and Electronics,
		University of Basque Country, UPV/EHU, {\tt\small raul.nistal@ehu.eus}},
		Manuel de la Sen\thanks{Manuel de la Sen, Department of Electricity and Electronics,
		University of Basque Country, UPV/EHU, {\tt\small manuel.delasen@ehu.eus}}, 
		Santiago Alonso-Quesada\thanks{Santiago Alonso-Quesada, Department of Electricity and Electronics,
		University of Basque Country, UPV/EHU, {\tt\small santiago.alonso@ehu.eus}}, 
		and Asier Ibeas\thanks{Asier Ibeas, Dept.of Telecommunications and Systems Engineering,
	 	Universitat Autonoma de Barcelona,{\tt\small  asier.ibeas@uab.cat}}  %% <-this % stops a space
		\thanks{This research is supported by the Spanish Government and by the European Fund of Regional Development FEDER through Grant DPI2015-64766-R and by UPV/EHU by Grant PGC 17/33}% <-this % stops a space
}
\title{On a discrete SEIADR model with Supervised control gains}
\begin{document}
\maketitle
A discrete SEIADR is built based on previous models we have worked with.
The model has four types of infected subpopulations.
Three of them are active individuals in three different states of the disease: Exposed individuals(E), symptomatically infectious individuals (E) and asymptomatically infectious(A), while the fourth one represents the individuals that have died from the disease but the corpses are still infected (D).
Thus, the objective is not only to treat the diseased but to retire the source of infection generated by the bodies.
The other subpopulations conforming our model will be the susceptible subpopulation (S) and the recovered subpopulation (R), which we will assume is has their immune system prepared to fight the disease and are unaffected by it.
A series of control gains are applied to the dynamic of the subpopulation in order to reduce the infected subpopulation down to zero.
These control strategies, such as the use of vaccination strategies or the medical treatment of the infectious subpopulation  has been proposed many times in the literature \cite{2,7,10,12,11,24,26,28,29}.\\
The control gains presented in this model will be:
\begin{list}{$\circ$}{}
	\item $V$, a constant vaccination applied to the new individuals that appear in the susceptible subpopulation at a rate $b_1$
	\item $K_V$, a proportional vaccination applied to the susceptible subpopulations, so that the rate of vaccination due to this control at a given time will be $K_V S(t)$.
	\item $K_\xi$, a proportional antiviral applied to the infectious subpopulation, so that the rate of cured individuals due to the treatment of the disease at a given time will be $K_\xi I(t)$.
\end{list}
Given this structure, we will build a discrete model based on a continuous one whose dynamic is defined by the following equations :
\begin{eqnarray}	
%	\begin{flushalign}
\dot{S}(t)&=&b_1 -V+\eta R(t)-S(t)\left(\frac{\beta I(t)+\beta_A A(t)+\beta_D D(t)}{N(t)}+b_2+K_V\right)\label{eqcontS}\\
\dot{E}(t)&=&S(t)\left(\frac{\beta I(t)+\beta_A A(t)+\beta_D D(t)}{N(t)}\right)-(b_2+\gamma)E(t)\\
\dot{I}(t)&=&p \gamma E -(b_2+\tau_0+K_\xi+\alpha)I(t)\\
\dot{A}(t)&=&(1-p)\gamma E-(b_2+\tau_0)A(t)\\
\dot{D}(t)&=&\alpha I(t)+(I(t)+A(t)b_2-\mu D(t)\\
\dot{R}(t)&=& V+K_\xi I(t)+K_V S(t)+\tau_0 (A(t)+I(t))-(b_2+\eta)R(t)\label{eqcontR}
%	\end{flushalign}
\end{eqnarray}
with $N(t)=S(t)+E(t)+I(t)+A(t)+D(t)+R(t)$ being the total population (plus the corpses), where S, E, I, A, D and R are the Susceptible, Exposed, Symptomatic Infectious, Asymptomatic Infectious, Dead Infectious corpses and Recovered subpopulations respectively.
The parameters of the model are:
\begin{itemize}
	\item{$b_1$} is the recruitment rate,
	\item{$b_2$} is the natural average death rate,
	\item{$ \beta,\beta_A,\beta_D$} are the various disease transmission coefficients from the susceptible to the respective symptomatic infectious, asymptomatic and infective corpses subpopulations,
	\item{$1/\eta$} is the average duration of the immunity period reflecting a transition from the recovered to the susceptible,
	\item{$\gamma$} is the transition rate from the exposed to all (i.e.both symptomatic and asymptomatic) the infectious,
	\item{$\alpha$} is the average extra mortality associated with the disease affecting to the symptomatic infectious subpopulation,
	\item{$\tau_0$} is the natural recovery rate for the whole infectious subpopulation (i.e., A + I ),
	\item{p} is the fraction of the exposed subpopulation which becomes symptomatic infectious,
	\item{1-p} is the fraction of the exposed subpopulation which becomes asymptomatic infectious,
	\item{1/ $\mu$ } is the average period of infectiousness after death,
\end{itemize}
\textbf{Remark 1} \\
We can rewrite the equations (\ref{eqcontS}-\ref{eqcontR}) with a set of auxiliary parameters to be then used as follows:\\\\
$B_E=b_2+\gamma$, $B_I=b_2+\tau_0+K_\xi+\alpha$, $B_A=b_2+\tau_0$, $B_R=b_2+\eta$,
$f=\frac{\gamma p }{B_I},$  $f_A=\frac{\gamma(1-p)}{B_A},$ $f_D=f_A\frac{b_2}{\mu}+f\frac{b_2+\alpha}{\mu},$ 
$G_1=B_R+B_E-b_2+\mu f_D \frac{B_R-b_2}{b_2}$, $F=f\beta+f_A\beta_A+f_D\beta_D$ and $G_2=(B_R+K_V)(b_2 f_D-\alpha f)$.\\\\
	After using these auxiliary parameters in the equations 1-6 the resulting equilibrium points of the solutions are :
\begin{itemize}
	\item 	A disease-free equilibrium point given by $(S_{dfe},0,0,0,0,R_{dfe})$ where
	\begin{eqnarray}
	S_{dfe}=\frac{B_R N_{dfe}-V}{B_R+K_V}\; ; \;\; R_{dfe}=\frac{K_V N_{dfe}+V}{B_R+K_V}\label{dfe-cont}
	\end{eqnarray}
	with  the total population at this point defined as $N_{dfe}=\frac{b_1}{b_2}$.
	\item
	An endemic one given by $(S_{end},E_{end},I_{end},A_{end},D_{end},R_{end})$ where:
	\begin{eqnarray}
	S_{end}=B_E\frac{G_2 S_{dfe}+b_2 G_1 N_{dfe}}{G_2 B_E+ b_2 F G_1},\;\; E_{end}=\frac{(S_{dfe}-S_{end})(B_R+K_V)}{G_1}\notag\\\notag\\
	I_{end}=f E_{end},\;\;A_{end}=f_A E_{end},\;\; D_{end}= f_D E_{end}\notag\\\notag\\
	R_{end}=R_{dfe}+\frac{(S_{dfe}-S_{end})(B_E-(b_2+\mu f_D (1+K_V/b_2)))}{G_1} \label{end-cont}
	\end{eqnarray}
	with the total population at this point given by:
	\begin{eqnarray}
	N_{end}=S_{end}+E_{end}+I_{end}+A_{end}+D_{end}+R_{end}=\frac{S_{end}F}{B_E}\notag
	\end{eqnarray}
\end{itemize}


\begin{eqnarray}	
%	\begin{flushalign}
\dot{S}(t)&=&b_1 -V+\eta R(t)-S(t)\left(\frac{\beta I(t)+\beta_A A(t)+\beta_D D(t)}{N(t)}+b_2+K_V\right)\label{eqcontS2}\\
\dot{E}(t)&=&B_E E(t)\left(\frac{\frac{S(t)\left(\beta I(t) +\beta_A A(t) +\beta_D D(t) \right)}{N(t)}}{\frac{S^*_{end}\left(\beta I^*_{end} +\beta_A A^*_{end} +\beta_D D^*_{end} \right)}{N^*_{end}}}-1\right)\\
\dot{I}(t)&=&B_I\left( f E(t)-I(t)\right)\\
\dot{A}(t)&=&B_A\left( f_A E(t)-A(t)\right)\\
\dot{D}(t)&=&\alpha I(t)+(I(t)+A(t)b_2-\mu D(t)\\
\dot{R}(t)&=& V+K_\xi I(t)+K_V S(t)+\tau_0 (A(t)+I(t))-(b_2+\eta)R(t)\label{eqcontR2}
%	\end{flushalign}
\end{eqnarray}
\bibliography{bib4}
\bibliographystyle{unsrt}
\end{document}